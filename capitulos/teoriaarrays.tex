%%%%%%%%%%%%%%%%%%%%%%%%%%%%%%%%%%%%%%%%%%%%%%%%%%%%%%%%%%%%%%%%%%%%%%%%
% Plantilla TFG/TFM
% Escuela Politécnica Superior de la Universidad de Alicante
% Realizado por: Jose Manuel Requena Plens
% Contacto: info@jmrplens.com / Telegram:@jmrplens
%%%%%%%%%%%%%%%%%%%%%%%%%%%%%%%%%%%%%%%%%%%%%%%%%%%%%%%%%%%%%%%%%%%%%%%%

\chapter{Arrays de Antenas}
\label{arraysdeantenas}

\section{Introducción}

\par Un array o vector de antenas se basa en la agrupación de un cierto número de elementos individuales de antenas para conseguir que trabajen como una sola, mejorando así las prestaciones globales de esta e incluso llegar a obtener parámetros característicos que serían imposibles de conseguir con el trabajo de una sola antena.

\par En cierto tipo de aplicaciones, necesitamos patrones de radiación o ganancias que antenas de un único elemento no nos pueden ofrecer, ya que los patrones de directividad de estos suelen ser anchos y muy poco directivos. Cuando se necesitan antenas con ganancias más elevadas o patrones de directividad estrechos y concentrados para, por ejemplo, radio enlaces, o patrones estrechos en un plano, pero muy extendidos en su perpendicular, como se hace necesario en aplicaciones de telefonía móvil donde se intenta concentrar el haz hacia las calles y las casas y evitar propagar la señal al cielo, se hace indispensable el uso de agrupaciones de antenas para modelar la radiación de la señal según nuestras especificaciones.
\\
\par Normalmente un array de antenas está formado por la agrupación del mismo tipo y modelo de antenas dispuestas en una geometría específica. Aunque esto no es estricto, y diferentes tipos de antenas pueden actuar como array para conseguir configuraciones más específicas o cuando otro tipo de limitaciones, como factores de diseño o económicos, no nos permiten usar el mismo tipo de antenas.
\\
\par Cuando se dispone de un array de antenas, lo común es que este se diseñe para que el campo electromagnético radiado total sea la suma de los campos individuales trabajando en forma de interferencia constructiva para el lóbulo principal, y en forma de interferencia destructiva para el resto del espacio, de forma que la máxima transferencia de potencia se centre en el lóbulo principal. Para conseguir un correcto funcionamiento de un array de antenas según las especificaciones marcadas para nuestra aplicación se deberán tener en cuenta una serie de factores clave en el diseño \cite{Valero2008}:

\begin{itemize}
\item \textbf{La configuración geométrica: }Se deberá conocer de antemano qué tipo de configuración geométrica se aplicará a la hora de distribuir los elementos individuales: Lineal, circular, rectangular, etc. Esta distribución es clave para la correcta aproximación a los parámetros finales deseados.

\item \textbf{Distancia entre elementos: }La distancia entre elementos  afectará a como los campos electromagnéticos interfieren entre si y estos son sumados en el campo lejano.

\item \textbf{La amplitud de excitación: }Una incorrecta intensidad de excitación a uno o varios de los elementos que conformen el array puede llevar a superposiciones indeseadas que deformen por completo el patrón de radiación de nuestra antena global.

\item \textbf{La fase de excitación: }La fase con la que se alimenta cada elemento individual es un factor clave para modelar el comportamiento global de la antena. Si se alimenta cada elemento siempre con la misma fase se obtendrá un patrón de radiación concreto y estático mientras que el hecho de variar la fase a cada elemento de forma individual hará que el patrón de radiación varíe según las necesidades. Esta variación de fase es normalmente llevada a cabo por ordenador y son también conocidos como \textit{Phased arrays}

\item \textbf{El patrón de radiación de cada elemento: }Para que un elemento llegue a interferir constructiva o destructivamente sobre la radiación de otro, es necesario que sus patrones lleguen a mezclarse físicamente, para ello, el patrón de radiación de cada elemento individual debe estar controlado para así poder predecir cual será el resultado de la antena global.
\end{itemize}

\section{Arrays lineales}
\par Para entender el funcionamiento analítico de las agrupaciones de antenas, se pondrá como ejemplo un array lineal de estas. Un array lineal consiste en un conjunto de antenas agrupadas a lo largo de una recta y conectadas en serie. El diagrama de radiación del array de antenas es el producto final de las interferencias constructivas y destructivas causadas por las radiaciones de los elementos individuales. Comenzaremos considerando una sola antena alimentada mediante una corriente $I_{n}$, donde $n$ identifica el número de la antena que compone el array, siendo 0 la primera antena y N-1 la última. Por lo tanto, la primera antena será alimentada por una corriente $I_{0}$ \citep{Cardama2002}.
\\
\par Esta antena poseerá una distribución de corrientes $J_{0}(\vec{r})$. Por lo tanto, si agrupamos un conjunto de $n$ antenas equiespaciadas una distancia $d$ a lo largo del eje $z$ (fig. \ref{fig:arraigo}), cada una excitada con su fasor de corriente, $I_{n}$, la distribución de corrientes del conjunto de antenas quedará como \cite{Cardama2002}:

\begin{figure}[h]
    \centering
        \includegraphics[width=0.8\textwidth]{archivos/array/array}
        \caption{Agrupación lineal de antenas sobre el eje Z}
        \label{fig:arraigo}
\end{figure}

\begin{equation}
	\vec{J}\left ( \vec{r} \right )=\sum_{n=0}^{N-1}I_{n}\vec{J}_{0}(r-nd\hat{z})
	\label{eq:distrib}
\end{equation}

\par Se puede expresar el sumatorio anterior como la convolución entre la corriente que alimenta un elemento básico del array, es decir, una antena simple, y un tren de deltas ponderadas con sus respectivos pesos $I_{n}$ \cite{Cardama2002}.

\begin{equation}
	\vec{J}\left ( \vec{r} \right )=\vec{J}_{0}\left ( \vec{r} \right ) \ast \sum_{n=0}^{N-1}I_{n}\delta (r-nd\hat{z})=\vec{J}_{0}\left ( \vec{r} \right )\ast I(n)
	\label{eq:conv}
\end{equation}

\par Sabiendo que el vector de radiación $\vec{N}\left ( \vec{r} \right )$, es la transformada de Fourier tridimensional de la distribución de corrientes $\vec{J}\left ( \vec{r} \right )$, se aplicará el teorema de convolución para calcularlo \cite{Cardama2002}.

\begin{equation}
	\vec{N}\left ( \vec{r} \right )=TF_{3D}\left [\vec{J}\left ( \vec{r} \right )  \right ]=\vec{N}_{0}\left ( \vec{r} \right )\cdot TF_{3D}\left [ I(n)   \right ]
	\label{eq:3d}
\end{equation}

\par Donde $\vec{N}_{0}\left ( \vec{r} \right )$ es el vector de radiación del elemento simple situado en el origen, cuando el fasor de alimentación toma el valor unidad. Dado que el fasor de corriente $I_{n}$  es separable, su $TF_{3D}$ consistirá en el producto de las transformadas en cada dirección \cite{Cardama2002}.


\begin{equation}
	TF_{3D}\left [ I(n)   \right ]= TF_{x}\left [ I(n)   \right ]\cdot TF_{y}\left [ I(n)   \right ]\cdot TF_{z}\left [ I(n)   \right ] = TF_{z}\left [ I(n)   \right ] = \sum_{n=0}^{N-1}I_{n} e^{j\omega _{z}n}
	\label{eq:direccional}
\end{equation}

\par Donde $\omega _{z}$ es la frecuencia digital en la dirección del eje $z$, que puede ser obtenida mediante el producto de la frecuencia espacial analógica $k_{z}$  por el periodo de muestreo en la dirección z, que equivale a la distancia de espaciación entre las antenas que componen el array, $d$ \cite{Cardama2002}. 

\begin{equation}
	\omega_{z} = k_{z}\cdot d = k d \cos {\theta } 
	\label{eq:frecdigital}
\end{equation}

\par Donde $\theta$ representa el ángulo cualquiera con respecto a la agrupación de antenas. Teniendo en cuenta que los fasores de alimentación $I_{n}$, presentan una fase progresiva entre cada par de antenas consecutivas que puede expresarse mediante \cite{Cardama2002}: 

\begin{equation}
	I_{n}=a_{n}e^{jn\alpha} 
	\label{eq:fasor}
\end{equation}

\par Donde $a_{n}$ son coeficientes, generalmente complejos, y que pueden tomar valores reales cuando la fase de alimentación sea progresiva, se podrá entonces
obtener el vector de radiación del conjunto de antenas \cite{Cardama2002}:

\begin{equation}
	\vec{N}\left ( \vec{r} \right )= \vec{N}_{0}\left ( \vec{r} \right )\sum_{n=0}^{N-1}a_{n}e^{jn(kd\cos\theta+\alpha)}
	\label{eq:vecrad}
\end{equation}

\par Para simplificar los cálculos, agruparemos el término $kd\cos\theta+\alpha$ en una sola variable, la cual representará la diferencia de fase entre las contribuciones en el campo lejano de dos antenas consecutivas \cite{Cardama2002}. 

\begin{equation}
	\Psi = kd\cos\theta+\alpha
	\label{eq:psi}
\end{equation}

\par Esta diferencia de fase es igual a la suma del desfase por diferencia de caminos $kd\cos\theta$, más la diferencia de fase que progresivamente ha ido alimentando cada antena $\alpha$. Quedando entonces el vector de radiación como \cite{Cardama2002}: 

\begin{equation}
	\vec{N}\left ( \vec{r} \right )= \vec{N}_{0}\left ( \vec{r} \right )\sum_{n=0}^{N-1}a_{n}e^{jn\Psi}
	\label{eq:vecrad2}
\end{equation}

\par Se puede observar cómo el vector de radiación consiste en el producto entre el vector de radiación de la primera antena básica $\vec{N}_{0}\left ( \vec{r} \right ) $ y un factor que tiene en cuenta la interferencia de las \textit{N} ondas generadas por cada antena. Este factor depende unicamente de la separación entre elementos, su alimentación y la frecuencia de trabajo, y se le denomina \textit{factor de agrupación} o \textit{factor de array} (\textit{FA} \cite{Cardama2002}.

\begin{equation}
	FA(\Psi)=\sum_{n=0}^{N-1}a_{n}e^{jn\Psi}
	\label{eq:fa}
\end{equation}

\section{Factor de array}

\par 