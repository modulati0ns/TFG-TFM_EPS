%%%%%%%%%%%%%%%%%%%%%%%%%%%%%%%%%%%%%%%%%%%%%%%%%%%%%%%%%%%%%%%%%%%%%%%%
% Plantilla TFG/TFM
% Escuela Politécnica Superior de la Universidad de Alicante
% Realizado por: Jose Manuel Requena Plens
% Contacto: info@jmrplens.com / Telegram:@jmrplens
%%%%%%%%%%%%%%%%%%%%%%%%%%%%%%%%%%%%%%%%%%%%%%%%%%%%%%%%%%%%%%%%%%%%%%%%
\thispagestyle{empty}
\vspace*{-20cm}
\chapter*{Resumen}

\par Las comunicaciones inalámbricas han supuesto una revolución en nuestra sociedad a la hora de comunicarnos e interactuar entre personas y máquinas. Uno de los elementos clave que hacen posible estas nuevas formas de comunicación son las antenas. Las antenas de tipo microstrip vienen siendo usadas desde los años ‘70 debido a su gran versatilidad y facilidad de diseño, integración, adaptación, fabricación, y transporte. Actualmente este tipo de antenas está a la orden del día y su uso en la nueva generación de comunicaciones móviles, el 5G, será crucial.
\\
\par En este proyecto de final de carrera se diseñarán una serie de antenas y conjunto de estas (arrays) de tipo parche en tecnología microstrip para distintas frecuencias como son: 2.4 GHz, banda actualmente usada internacionalmente para la transmisión de información mediante tecnologías Wi-Fi y Bluetooth, 6 GHz, banda actualmente reservada en España para el uso en radiocomunicaciones vía satélite y entre estaciones de comunicación fijas, y cuya posible utilización para el despliegue de redes 5G y nuevas tecnologías Wi-Fi está siendo estudiado, y 27 GHz, banda ya reservada para el uso en la 5ª generación de comunicaciones móviles.
\\
\par Durante el proyecto se hará uso de la bibliografía existente en materia de antenas en tecnología microstrip, en especial la teoría recopilada por Constantine A. Balanis en su libro ``Antenna Theory" y Rod Waterhouse en su libro ``Microstrip Patch Antennas: A Designer's Guide", donde se recogen las principales ecuaciones necesarias para el correcto diseño de antenas microstrip y el conjunto de estas. Así mismo se hará uso de otros libros y artículos que han servido de ayuda para darle forma y sentido al proyecto.
\\
\par Para el diseño y desarrollo de los arrays de antenas tipo parche en tecnología microstrip se hará uso de dos herramientas principales: MATLAB para el desarrollo de \textit{scripts} de cálculo y la obtención de los datos necesarios para el correcto diseño de las antenas, y Ansys HFSS, para la implementación y simulación de la antena, una herramienta de análisis electromagnético profesional que facilitará la correcta comprensión de los resultados obtenidos y brindará todas las funciones necesarias para conocer la forma en la que la antena diseñada se comporta. Para cada frecuencia se realizarán seis diseños diferentes, desde un parche simple, hasta un array de 4x4 parches microstrip, excepto para el caso a 27 GHz, donde solo se llegará hasta el array 2x2 puesto que las limitaciones de diseño (alimentación, etc.) hacen que no sea útil el diseño de arrays tan complejos para el uso que las antenas a esta frecuencia pueden tener.
\\
\par Tras el análisis de diseños y parámetros principales de cada antena, se realizará una conclusión sobre el proyecto en la que se tendrán en cuenta qué factores han sido claves para la limitación de la efectividad de los resultados obtenidos, así como recomendaciones para futuras líneas de trabajo relacionadas con el diseño de antenas microstrip y posibles nuevas características de las que se pueden dotar, como polarización circular o nuevos métodos de alimentación.





\cleardoublepage %salta a nueva página impar
\chapter*{Agradecimientos}

\thispagestyle{empty}
\vspace{1cm}

\par Este trabajo de final de grado pone punto y seguido a mis cuatro años como estudiante de ingeniería de telecomunicaciones, y es aquí donde me es imposible no mirar atrás y observar con vértigo hasta donde he podido llegar.
Pero todo este camino ha tenido muchos nombres y apellidos a los que me gustaría agradecer su apoyo y confianza:

\par A mis padres, Javier Martínez y Rosa Manzano, mi hermano, Alejando y toda mi familia, cuyo cariño, apoyo y confianza desde 1996 han hecho que pueda llegar hasta donde hoy me encuentro.



\par A todos aquellos docentes que, desde infantil hasta la universidad, me han servido como referentes, aportándome nuevas formas de ver el mundo y haciéndome ver cuál es mi camino en la vida: la tecnología y las telecomunicaciones.




\par A Stephan Marini y Miguel Angel Sanchez Soriano, mis dos tutores, por brindarme la oportunidad de realizar este proyecto sobre la materia que más me apasiona: las antenas, y haberme ofrecido en todo momento su ayuda y conocimientos.




\par A mis compañeros de clase por haber hecho que estos cuatro años de universidad hayan sido un poco más fáciles. En especial a Quique, cuyo apoyo y compañía durante todo este tiempo llevaré para el resto de mi vida, y a Plens, porque su dedicación y sus ganas de ayudar a los demás han marcado la forma de ser de una generación entera de \textit{telecos}, y gracias a ello este TFG va a quedar precioso. 




\par A Pablo Mateos por toda su ayuda para este TFG. Quién diría que cuatro años después de que la selectividad nos separara, unos simples elementos radiantes nos harían reencontrarnos.




\par Y a Anaida, por literalmente: TODO. 
\cleardoublepage %salta a nueva página impar
% Aquí va la dedicatoria si la hubiese. Si no, comentar la(s) linea(s) siguientes
\chapter*{}
\setlength{\leftmargin}{0.5\textwidth}
\setlength{\parsep}{0cm}
\addtolength{\topsep}{0.5cm}
\begin{flushright}
\small\em{
Wio, alguien en una terraza ha gritado, "Te amo"\\
Una suave interferencia, culpa al viento solar\\
Un poema embotellado que en estéreo ha aterrizado en mi inconsciente.\\}
\end{flushright}


\begin{flushright}
\small{
"Wio, Antenas y Pijamas" - Love Of Lesbian
}
\end{flushright}


\cleardoublepage %salta a nueva página impar

