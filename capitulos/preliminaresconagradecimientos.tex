%%%%%%%%%%%%%%%%%%%%%%%%%%%%%%%%%%%%%%%%%%%%%%%%%%%%%%%%%%%%%%%%%%%%%%%%
% Plantilla TFG/TFM
% Escuela Politécnica Superior de la Universidad de Alicante
% Realizado por: Jose Manuel Requena Plens
% Contacto: info@jmrplens.com / Telegram:@jmrplens
%%%%%%%%%%%%%%%%%%%%%%%%%%%%%%%%%%%%%%%%%%%%%%%%%%%%%%%%%%%%%%%%%%%%%%%%

\chapter*{Resumen}
\thispagestyle{empty}
\par Las comunicaciones inalámbricas han supuesto una revolución en nuestra sociedad a la hora de comunicarnos e interactuar entre personas y máquinas. Uno de los elementos clave que hacen posible estas nuevas formas de comunicación son las antenas. Las antenas de tipo microstrip vienen siendo usadas desde los años ‘70 debido a su gran versatilidad y facilidad de diseño, integración, adaptación, fabricación, y transporte. Actualmente este tipo de antenas está a la orden del día y su uso en la nueva generación de comunicaciones móviles, el 5G, será crucial.
\\
\par En este proyecto de final de carrera se diseñarán una serie de antenas y conjunto de estas (arrays) de tipo parche en tecnología microstrip para distintas frecuencias como son: 2.4 GHz, banda actualmente usada internacionalmente para la transmisión de información mediante tecnologías Wi-Fi y Bluetooth.  6 GHz, banda actualmente reservada en España para el uso en radiocomunicaciones vía satélite y entre estaciones de comunicación fijas, y cuya posible utilización para el despliegue de redes 5G está siendo estudiado, y 27 GHz, banda ya reservada para el uso en la 5ª generación de comunicaciones móviles.





\cleardoublepage %salta a nueva página impar
\chapter*{Agradecimientos}

\thispagestyle{empty}
\vspace{1cm}

\par Este trabajo de final de grado pone punto y seguido a mis cuatro años como estudiante de ingeniería de telecomunicaciones, y es aquí donde me es imposible no mirar atrás y observar con vértigo hasta donde he podido llegar.
Pero todo este camino ha tenido muchos nombres y apellidos a los que me gustaría agradecer su apoyo y confianza:

\par A mis padres, Javier Martínez y Rosa Manzano, mi hermano, Alejando y toda mi familia, cuyo cariño, apoyo y confianza desde 1996 han hecho que pueda llegar hasta donde hoy me encuentro.



\par A todos aquellos docentes que, desde infantil hasta la universidad, me han servido como referentes, aportándome nuevas formas de ver el mundo y haciéndome ver cuál es mi camino en la vida: la tecnología y las telecomunicaciones.




\par A Stephan Marini y Miguel Angel Sanchez Soriano, mis dos tutores, por brindarme la oportunidad de realizar este proyecto sobre la materia que más me apasiona: las antenas, y haberme ofrecido en todo momento su ayuda y conocimientos.




\par A mis compañeros de clase por haber hecho que estos cuatro años de universidad hayan sido un poco más fáciles. En especial a Quique, cuyo apoyo y compañía durante todo este tiempo llevaré para el resto de mi vida, y a Plens, porque su dedicación y sus ganas de ayudar a los demás han marcado la forma de ser de una generación entera de \textit{telecos}, y gracias a ello este TFG va a quedar precioso. 




\par A Pablo Mateos por toda su ayuda para este TFG. Quién diría que cuatro años después de que la selectividad nos separara fueras a elegir el mismo camino que yo.




\par Y a Anaida, por literalmente: TODO. 
\cleardoublepage %salta a nueva página impar
% Aquí va la dedicatoria si la hubiese. Si no, comentar la(s) linea(s) siguientes
\chapter*{}
\setlength{\leftmargin}{0.5\textwidth}
\setlength{\parsep}{0cm}
\addtolength{\topsep}{0.5cm}
\begin{flushright}
\small\em{
Wio, alguien en una terraza ha gritado, "Te amo"\\
Una suave interferencia, culpa al viento solar\\
Un poema embotellado que en estéreo ha aterrizado en mi inconsciente,\\}
\end{flushright}


\begin{flushright}
\small{
"Wio, Antenas y Pijamas" - Love Of Lesbian
}
\end{flushright}


\cleardoublepage %salta a nueva página impar

