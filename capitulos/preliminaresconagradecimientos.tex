%%%%%%%%%%%%%%%%%%%%%%%%%%%%%%%%%%%%%%%%%%%%%%%%%%%%%%%%%%%%%%%%%%%%%%%%
% Plantilla TFG/TFM
% Escuela Politécnica Superior de la Universidad de Alicante
% Realizado por: Jose Manuel Requena Plens
% Contacto: info@jmrplens.com / Telegram:@jmrplens
%%%%%%%%%%%%%%%%%%%%%%%%%%%%%%%%%%%%%%%%%%%%%%%%%%%%%%%%%%%%%%%%%%%%%%%%


\cleardoublepage %salta a nueva página impar
\chapter*{Agradecimientos}

\thispagestyle{empty}
\vspace{1cm}

\par Este trabajo de final de grado pone punto y seguido a mis cuatro años como estudiante de telecomunicaciones, y es aquí donde me es imposible no mirar atrás y observar con vértigo hasta donde he podido llegar.
Pero todo este camino ha tenido muchos nombres y apellidos a los que me gustaría agradecer su apoyo y confianza:

\par A mis padres, Javier Martínez y Rosa Manzano, mi hermano, Alejando y toda mi familia, cuyo cariño, apoyo y confianza desde 1996 han hecho que pueda llegar hasta donde hoy me encuentro.



\par A todos aquellos docentes que, desde infantil hasta la universidad, me han servido como referentes, aportándome nuevas formas de ver el mundo y haciéndome ver cual es mi camino en la vida: La tecnología y las telecomunicaciones.




\par A Stephan Marini y Miguel Angel Sanchez Soriano, mis dos tutores, por brindarme la oportunidad de realizar este proyecto sobre la materia que más me apasiona: Las antenas, y haberme ofrecido en todo momento su ayuda y conocimientos.




\par A mis compañeros de clase por haber hecho que estos cuatro años de clase hayan sido un poco más fáciles. En especial a Quique, cuyo apoyo y compañía durante estos cuatro años llevaré para el resto de mi vida, y a Plens, porque su dedicación y sus ganas de ayudar a los demás han marcado la forma de ser de una generación entera de telecos, y gracias a ello este TFG va a quedar precioso. 




\par A Pablo Mateos por todos su ayuda para este TFG. Quién diría de cuatro años después de que la selectividad nos separara fueras a elegir el mismo camino que yo.




\par Y a Anaida, por literalmente: TODO. 
\cleardoublepage %salta a nueva página impar
% Aquí va la dedicatoria si la hubiese. Si no, comentar la(s) linea(s) siguientes
\chapter*{}
\setlength{\leftmargin}{0.5\textwidth}
\setlength{\parsep}{0cm}
\addtolength{\topsep}{0.5cm}
\begin{flushright}
\small\em{
Wio, alguien en una terraza ha gritado, "Te amo"\\
Una suave interferencia, culpa al viento solar\\
Un poema embotellado que en estéreo ha aterrizado en mi inconsciente,\\}
\end{flushright}


\begin{flushright}
\small{
"Wio, Antenas y Pijamas" - Love Of Lesbian
}
\end{flushright}


\cleardoublepage %salta a nueva página impar

