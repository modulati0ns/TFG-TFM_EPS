%%%%%%%%%%%%%%%%%%%%%%%%%%%%%%%%%%%%%%%%%%%%%%%%%%%%%%%%%%%%%%%%%%%%%%%%
% Plantilla TFG/TFM
% Escuela Politécnica Superior de la Universidad de Alicante
% Realizado por: Jose Manuel Requena Plens
% Contacto: info@jmrplens.com / Telegram:@jmrplens
%%%%%%%%%%%%%%%%%%%%%%%%%%%%%%%%%%%%%%%%%%%%%%%%%%%%%%%%%%%%%%%%%%%%%%%%

\chapter{Diseño y análisis de arrays de parches microstrip}
\label{diseño}

\section{Introducción}

\par En este capítulo se va a abordar la realización del diseño inicial de un conjunto de configuraciones de arrays de antenas microstrip para las frecuencias de 2.4 Ghz, 6 Ghz y 27 Ghz. Como se mencionó en la sección \ref{sec:medodologia}, el diseño y simulación de estas configuraciones se realizará mediante la herramienta Ansys \sffamily\textregistered  HFSS y MathWorks \sffamily\textregistered  MATLAB.
\\
\par Todos las antenas de parche microstrip individuales y las posibles configuraciones de array que se diseñen con el conjunto de estas, estarán basadas en unos criterios y especificaciones de construcción común:

\begin{itemize}
	\item \textbf{Polarización: }Lineal
	\item \textbf{Tipo de alimentación: Directa por línea microstrip}
	\item \textbf{Impedancia de entrada: }50 $\Omega$
	\item \textbf{Altura del substrato: }1.52 \SI{3.55}{\milli\metre}
	\item \textbf{Altura de los planos conductores: }\SI{3.55}{\micro\metre}
	\item \textbf{Substrado: }Rogers 4003 (RO4003)
	\item \textbf{Constante dieléctrica del substrato: }3.55
\end{itemize}

\section{Cálculos iniciales con MATLAB}
\par En primer lugar, se realizará un síntesis de las ecuaciones necesarias para diseñar parches microstrip en MATLAB. Para ello usaremos las ecuaciones recogidas en la sección \ref{analisis}. Se explicará paso a paso el código implementado para la obtención de estos resultados.
\\
\par En primer lugar se realizará la declaración inicial de variables, donde se almacenarán en la memoria del computador los valores de las constantes que se usarán a lo largo de los cálculos. La única variable que tendrá que ser introducida a mano por el usuario es la frecuencia en Ghz a la que se desea realizar el diseño de la antena. Además, desde aquí se realizará el cálculo de otros parámetros que necesitaremos más adelante, como la longitud de onda \textit{$\lambda_{0}$} o el número de onda \textit{k}.

\begin{lstlisting}[style=Matlab-color, caption={Declaración de variables iniciales},label=variniciales]
%% Input Variables
f = input('Introduzca la frecuencia de trabajo deseada (Ghz): ');   % Frecuencia a la que se vana a realizar los cálculos

er = 3.55;										% Constante dieléctrica
h = 1.52;										% Altura del substrato

% Otras variables
c = physconst('LightSpeed');					% Velocidad de la luz
f = f*1e9;										% Frecuencia en Ghz
h = h*1e-3;										% Longitud en mm
lambda = c/f;									% Longitud de onda en el vacío
ko = 2*pi/lambda;								% Número de onda
Zo = 50;										% Impedancia de entrada
\end{lstlisting}

\par A continuación se procederán calcular los parámetros característicos del diseño de la antena como son su anchura \textit{W}, longitud \textit{L}, etc. 

\begin{lstlisting}[style=Matlab-color, caption={Parámetros de diseño de la antena},label=diseñoantena]
%% Cálculos del Parche

W = (c/(2*f))*sqrt(2/(er+1));                       % Ancho del Parche (Width)
erff = ((er+1)/2) + ((er-1)/2)*(1+12*h/W)^(-1/2);   % Coeficiente del dieléctrico efectiva
Leff = c/(2*f*sqrt(erff));                          % Longitud efectiva
Al = ((0.412*h*(erff+0.3)*((W/h)+0.264))/((erff-0.258)*((W/h)+0.8))); % Incremento de Longitud normalizada
L = Leff - 2*Al;                                    % Longitud del parche
a = 0.7*lambda;                                     % Separacion entre parches
\end{lstlisting}

\par También calcularemos la anchura $W_{feed}$ y longitud $L_{feed}$ necesaria para las líneas de transmisión microstrip así como la longitud para las líneas que actúen como transformadores $\lambda /4$. La anchura de las líneas microstrip será calculada en base a la variable $Z_{0}$, la cual ha sido declarada anteriormente con un valor de 50 ($\Omega$), este valor puede ser cambiado en el código para obtener la anchura de la línea para sintetizar cualquier impedancia que necesitemos en esta.

\begin{lstlisting}[style=Matlab-color, caption={Parámetros de diseño de la línea de alimentación},label=alimentacion]
%% Cálculos de línea de alimentación

lambdaguided = lambda/sqrt(erff);               % Longitud de onda en medio guiado
Lfeed = lambdaguided/4;                         % Longitud lambda cuartos

%Calculo de la anchura de la línea
A = (Zo/60)*(sqrt((er+1)/2))+((er-1)/(er+1))*(0.23+(0.11/er)); 
B = (377*pi)/(2*Zo*sqrt(er));
Coef = (8*exp(A))/((exp(2*A))-2);
if (Coef <= 2)
	Wline = Coef*h;                             % Anchura si W/h <= 2
elseif (Coef > 2)
    Coef = (2/pi)*( B-1-log(2*B-1)+((er-1)/(2*er))*(log(B-1)+0.39-(0.61/er)));
	Wline = Coef*h;                             % Anchura si W/h > 2
end
\end{lstlisting}

\par Finalmente, se calculará la longitud que deberán tener las ranuras o \textit{insets} para adaptar que la línea de alimentación encuentre la posición dentro de la antena donde se adaptan sus impedancias.

\begin{lstlisting}[style=Matlab-color, caption={Parámetros de diseño del \textit{inset}},label=inset]
%% Inset

I1 = @(theta) (sin((ko*W/2)*cos(theta))./cos(theta)).^2.*sin(theta).^3;
G1 = integral(I1,0,pi)/(120*pi^2);          % Admitancia de la TL
I2 = @(theta) ((sin((ko*W/2)*cos(theta))./cos(theta)).^2).*besselj(0,ko*L*sin(theta)).*sin(theta).^3;
G12 = (1/(120*pi^2)).*integral(I2,0,pi);        % Admitancia mutua
Rin = 1./(2*(G1+G12));                          % Impedancia de entrada
yo = (L/pi).*acos(sqrt(Zo/Rin));        % Longitud del inset
\end{lstlisting}

\par Con todos estos parámetros calculados se podrá proceder a diseñar y analizar los arrays de antenas en HFSS.

\section{Diseño y Análisis en Ansys HFSS}



