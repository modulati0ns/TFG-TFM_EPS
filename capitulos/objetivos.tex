%%%%%%%%%%%%%%%%%%%%%%%%%%%%%%%%%%%%%%%%%%%%%%%%%%%%%%%%%%%%%%%%%%%%%%%%
% Plantilla TFG/TFM
% Escuela Politécnica Superior de la Universidad de Alicante
% Realizado por: Jose Manuel Requena Plens
% Contacto: info@jmrplens.com / Telegram:@jmrplens
%%%%%%%%%%%%%%%%%%%%%%%%%%%%%%%%%%%%%%%%%%%%%%%%%%%%%%%%%%%%%%%%%%%%%%%%

\chapter{Antenas Microstrip}
\label{objetivos}

\section{Introducción}
\par Las antenas de parche o antenas microstrip son un tipo de antenas de tipo planar que utilizan la tecnología microstrip para su funcionamiento. Este tipo de tecnología fue estudiada por primera vez en la década de 1950, pero fue dos décadas más tarde, en 1970 cuando, gracias al desarrollo de los \gls{pcb}, se pudieron empezar a realizar los primeros desarrollos de líneas de transmisión y antenas con tecnología microstrip. Desde entonces, las antenas microstrip se han convertido en uno de los tipos de antena más usado para un alto abanico de aplicaciones. 
\\
\par Entre sus principales ventajas se encuentra su delgadez y capacidad de adaptación a distintos tipos de superficies, incluso pudiendo ser conformadas en superficies curvas y no planares. Además son antenas simples, muy ligeras, fáciles de diseñar, con un coste de producción bajo, fáciles de transportar, y preparadas para ser integradas en arrays. Por estas razones, los circuitos y antenas microstrip son comúnmente usados para la fabricación de circuitos monolíticos integrados para microondas (MMICs) en aplicaciones civiles, militares, gubernamentales y comerciales como identificación por radio frecuencia (RFID), retransmisión de radio, sistemas de comunicaciones móviles, \gls{gps}, televisión, comunicaciones por satélite, sistemas de vigilancia, radar, y guiado de misiles entre otros. 
\\
\par Por otra parte, las antenas microstrip son muy versátiles en términos de resonancia y polarización, con obtención de buenos patrones de radiación y fácil adaptación de impedancias. Si además sumamos al circuito elementos adaptativos como diodos varicap, se pueden llegar a diseñar antenas microstrip con frecuencias, impedancias, y patrones de radiación variables. En contra, el uso de antenas microstrip conlleva ciertas limitaciones como su alto factor de calidad (Q), necesidad de limitar la potencia que atraviesa el circuito, baja eficiencia, baja pureza de polarización y ancho de banda limitado. En ciertas aplicaciones, estas limitaciones pueden ser usadas a favor, como el hecho de tener bajos anchos de banda puede ser deseado a la hora de usar las antenas microstrip en aplicaciones de seguridad gubernamental puesto que se limita el rango de penetración externa por parte de posibles atacantes.









