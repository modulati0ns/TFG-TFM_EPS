%%%%%%%%%%%%%%%%%%%%%%%%%%%%%%%%%%%%%%%%%%%%%%%%%%%%%%%%%%%%%%%%%%%%%%%%
% Plantilla TFG/TFM
% Escuela Politécnica Superior de la Universidad de Alicante
% Realizado por: Jose Manuel Requena Plens
% Contacto: info@jmrplens.com / Telegram:@jmrplens
%%%%%%%%%%%%%%%%%%%%%%%%%%%%%%%%%%%%%%%%%%%%%%%%%%%%%%%%%%%%%%%%%%%%%%%%

\chapter{Conclusiones y líneas futuras}
\label{conclusiones}

\par Para finalizar este proyecto de final de carrera, se mencionarán las principales conclusiones obtenidas a lo largo del desarrollo del proyecto así como unas posibles recomendaciones o líneas futuras aplicables a cualquier persona que, en un futuro, desee seguir la senda del desarrollo de antenas microstrip y arrays de estas.

\par En primer lugar, veo necesario volver a mencionar la importancia de la tecnología trabajada durante el proyecto en la sociedad, y aun más importante en la nueva década que va a dar comienzo en breves. Los seres humanos se encuentran en un constante proceso de comunicación con las personas y el medio que los rodea. Este proceso, ha ido evolucionando y mejorando con el paso del tiempo hasta el día de hoy, donde nos es imposible imaginar una vida sin los beneficios que nos ha ofrecido la tecnología tanto en materia de comunicación como en productividad, entretenimiento o seguridad. En la mayoría de los casos ya tenemos asumida la integración de la tecnología en nuestra vida, por lo que no vemos los elementos que la hacen posible, y es aquí donde hago hincapié en resaltar la importancia de las antenas en todo este proceso, los verdaderos emisores y receptores del proceso de comunicación electrónico. 
\\
\par El proceso de diseño y desarrollo de nuevas antenas no ha cesado desde la invención de Marconi en 1895 hasta el día de hoy, donde grandes empresas dedican sus esfuerzos en diseñar antenas más eficientes, de bajo consumo e impacto ambiental, inteligentes, y con capacidad de brindar el proceso de comunicación a miles de personas a la vez. Entre las muchas tecnologías diferentes de antenas, las tratadas en este proyecto, las antenas microstrip, han sido desde su invención en la década de 1950, una de las tecnologías más importantes gracias a su versatilidad, eficiencia, precio de fabricación, tamaño y prestaciones, y su uso seguirá vigente en las nuevas generaciones de comunicación como es el \gls{5g}, cuyo despliegue de infraestructura está previsto para completarse entre 2020 y 2021. 
\\
\par Aunque en esta nueva generación de comunicaciones móviles, muchas de las bases y especificaciones ya hayan sido establecidas, queda por llevar a cabo el proceso de adaptación de la tecnología de radio comunicaciones existente a las nuevas especificaciones requeridas por el estándar. Es aquí donde este proyecto de final de carrera intenta aportar su granito de arena mediante el diseño de antenas capaces de funcionar a frecuencias como los 27 GHz previstos para esta nueva generación móvil, así como otras frecuencias útiles en tecnologías ya establecidas como el \gls{wifi}.
\\
\par En cuanto a las antenas diseñadas, y como se ha podido comprobar en los capítulos \ref{diseño} y \ref{resultados} han ido variando desde antenas simples hasta arrays completos que ofrecían prestaciones muy competentes. 