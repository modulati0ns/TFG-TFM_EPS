%%%%%%%%%%%%%%%%%%%%%%%%%%%%%%%%%%%%%%%%%%%%%%%%%%%%%%%%%%%%%%%%%%%%%%%%
% Plantilla TFG/TFM
% Escuela Politécnica Superior de la Universidad de Alicante
% Realizado por: Jose Manuel Requena Plens
% Contacto: info@jmrplens.com / Telegram:@jmrplens
%%%%%%%%%%%%%%%%%%%%%%%%%%%%%%%%%%%%%%%%%%%%%%%%%%%%%%%%%%%%%%%%%%%%%%%%

% Lista de acrónimos (se ordenan por orden alfabético automáticamente)

% La forma de definir un acrónimo es la siguiente:
% \newacronym{id}{siglas}{descripción}
% Donde:
% 	'id' es como vas a llamarlo desde el documento.
%	'siglas' son las siglas del acrónimo.
%	'descripción' es el texto que representan las siglas.
%
% Para usarlo en el documento tienes 4 formas:
% \gls{id} - Añade el acrónimo en su forma larga y con las siglas si es la primera vez que se utiliza, el resto de veces solo añade las siglas. (No utilices este en títulos de capítulos o secciones).
% \glsentryshort{id} - Añade solo las siglas de la id
% \glsentrylong{id} - Añade solo la descripción de la id
% \glsentryfull{id} - Añade tanto  la descripción como las siglas

\newacronym{ieee}{IEEE}{Institute of Electrical and Electronics Engineers}
\newacronym{csic}{CSIC}{Consejo Superior de Investigaciones Científicas}
\newacronym{3gpp}{3GPP}{3rd Generation Partnership Project}
\newacronym{1g}{1G}{1ª Generación}
\newacronym{2g}{2G}{2ª Generación}
\newacronym{3g}{3G}{3ª Generación}
\newacronym{5g}{5G}{5ª Generación}
\newacronym{4g}{4G}{4ª Generación}
\newacronym{4glte}{4G LTE}{4ª Generación (Long Term Evolution)}
\newacronym{lte}{LTE}{Long Term Evolution}
\newacronym{voip}{VoIP}{Voice over IP}
\newacronym{ran}{RAN}{Radio Access Network}
\newacronym{ntt}{NTT}{Nippon Telegraph and Telephone}
\newacronym{gsm}{GSM}{Global System for Mobile Communications}
\newacronym{gprs}{GPRS}{General Packet Radio Service}
\newacronym{nmt}{NMT}{Nordic Mobile Telephone}
\newacronym{amps}{AMPS}{Advanced Mobile Phone System}
\newacronym{tacs}{TACS}{Total Access Communications System}
\newacronym{tma}{TMA}{Telefonía Móvil Automática}
\newacronym{ctne}{CTNE}{Compañía Telefónica Nacional de España}
\newacronym{iot}{IoT}{Internet Of Things}
\newacronym{m2m}{M2M}{Machine To Machine}
\newacronym{mimo}{MIMO}{Multiple Imput Multiple Output}
\newacronym{wifi}{Wi-Fi}{}
\newacronym{tic}{TIC}{Tecnologías de la Información y la Comunicación}
\newacronym{mmimo}{mMIMO}{Massive MIMO}
\newacronym{tdt}{TDT}{Televisión Digital Terrestre}
\newacronym{minetur}{MINETUR}{Ministerio de Industria, Comercio y Turismo}
\newacronym{mmwv}{mmWave}{Milimeter Wave}
\newacronym{oem}{OEM}{Ondas Electromagnéticas}
\newacronym{itu}{ITU}{International Telecommunication Union}
\newacronym{umts}{UMTS}{Universal Mobile Telecommunications System}
\newacronym{att}{ATT}{American Telephone and Telegraph}
\newacronym{wcdma}{W-CDMA}{Wideband Code Division Multiple Access}
\newacronym{hspa}{HSPA}{High Speed Packet Access}
\newacronym{ofdm}{OFDM}{Orthogonal Frequency Division Multiplexing}
\newacronym{qam}{QAM}{Quadrature Amplitude Modulation}
\newacronym{fm}{FM}{Frecuencia Modulada}
\newacronym{am}{AM}{Amplitud Modulada}
\newacronym{te}{TE}{Transversal Eléctrico}
\newacronym{tm}{TM}{Transversal Magnético}
\newacronym{gtd}{GTD}{Geometric Theory of Diffraction}
\newacronym{iram}{IRAM}{Instituto Radio Astronómico Milimétrico}
\newacronym{gps}{GPS}{Global Positioning System}
\newacronym{pcb}{PCB}{Printed Circuit Board}
\newacronym{fem}{FEM}{Finite Element Method}
