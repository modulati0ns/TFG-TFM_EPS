%%%%%%%%%%%%%%%%%%%%%%%%%%%%%%%%%%%%%%%%%%%%%%%%%%%%%%%%%%%%%%%%%%%%%%%%
% Plantilla TFG/TFM
% Escuela Politécnica Superior de la Universidad de Alicante
% Realizado por: Jose Manuel Requena Plens
% Contacto: info@jmrplens.com / Telegram:@jmrplens
%%%%%%%%%%%%%%%%%%%%%%%%%%%%%%%%%%%%%%%%%%%%%%%%%%%%%%%%%%%%%%%%%%%%%%%%


% Ejemplo de páginas en horizontal y vertical
\chapter{Anexo I - Códigos}
\label{codigos}


\section{Calculador de dimensiones del parche}
\par Este es el código fuente completo escrito en MATLAB diseñado para calcular las dimensiones necesarias tanto para el parche como otros cálculos que han sido necesarios para el diseño de los distintos arrays de parches en tecnología microstrip. 
\\
\begin{lstlisting}[style=Matlab-color, caption={\textit{AntennaPatchCalculator.m}},label=codigofull]
%% CALCULADOR DE PARAMETROS Y DIMENSIONES DE ANTENAS DE PARCHE MICROSTRIP
%  Javier Martínez Manzano - Trabajo de Fin de Grado - UA - 2018/19
clear;
clc;
%% Input Variables
f = input('Introduzca la frecuencia de trabajo deseada (Ghz): ');   % Frecuencia a la que se vana a realizar los cálculos

er = 3.55;                                      % Constante dieléctrica
h = 1.52;                                       % Altura del substrato

% Otras variables
c = physconst('LightSpeed');                    % Velocidad de la luz
f = f*1e9;                                      % Frecuencia en Ghz
h = h*1e-3;                                     % Longitud en mm
lambda = c/f;                                   % Longitud de onda en el vacío
ko = 2*pi/lambda;                               % Número de onda
Zo = 70.71;                                        % Impedancia de entrada

%% Cálculos del Parche

W = (c/(2*f))*sqrt(2/(er+1));                       % Ancho del Parche (Width)
erff = ((er+1)/2) + ((er-1)/2)*(1+12*h/W)^(-1/2);   % Coeficiente del dieléctrico efectiva
Leff = c/(2*f*sqrt(erff));                          % Longitud efectiva
Al = ((0.412*h*(erff+0.3)*((W/h)+0.264))/((erff-0.258)*((W/h)+0.8))); % Incremento de Longitud normalizada
L = Leff - 2*Al;                                    % Longitud del parche
a = 0.7*lambda;                                     % Separacion entre parches

%% Cálculos del dieléctrico

g = 0.0606*lambda/sqrt(er);                     % Variable auxiliar
Lsub = L + 6*g;                                 % Longitud del substrato
Wsub = W + 6*g;                                 % Ancho del substrato


%% Cálculos de línea de alimentación

lambdaguided = lambda/sqrt(erff);               % Longitud de onda en medio guiado
Lfeed = lambdaguided/4;                         % Longitud lambda cuartos

%Calculo de la anchura de la línea
for Zo = [25, 50, 70.71, 100]
A = (Zo/60)*(sqrt((er+1)/2))+((er-1)/(er+1))*(0.23+(0.11/er)); 
B = (377*pi)/(2*Zo*sqrt(er));
Coef = (8*exp(A))/((exp(2*A))-2);
if (Coef <= 2)
    Wline = Coef*h;                             % Anchura si W/h <= 2
elseif(Coef > 2)
    Coef = (2/pi)*( B-1-log(2*B-1)+((er-1)/(2*er))*(log(B-1)+0.39-(0.61/er)));
    Wline = Coef*h;                             % Anchura si W/h > 2
end
disp(['Ancho de pista a ',num2str(Zo),': ',num2str(Wline*(1e3)), ' mm']);
end
%% Inset

I1 = @(theta) (sin((ko*W/2)*cos(theta))./cos(theta)).^2.*sin(theta).^3;
G1 = integral(I1,0,pi)/(120*pi^2);          % Admitancia de la TL
I2 = @(theta) ((sin((ko*W/2)*cos(theta))./cos(theta)).^2).*besselj(0,ko*L*sin(theta)).*sin(theta).^3;
G12 = (1/(120*pi^2)).*integral(I2,0,pi);        % Admitancia mutua
Rin = 1./(2*(G1+G12));                          % Impedancia de entrada
yo = (L/pi).*acos(sqrt(100/Rin));        % Longitud del inset



%% Caja de Radiación

%Lrad = lambdaguided/6 + lambdaguided/6 + Lsub;  % Longitud de la Rad Box
%Wrad = lambdaguided/6 + lambdaguided/6 + Wsub;  % Ancho de la Rad Box
%Hrad = lambdaguided/6 + lambdaguided/6 + h;  % Ancho de la Rad Box

%% Resultados:

disp(['Patch Width:',num2str(W*(1e3)), ' mm']);
disp(['Patch Length:',num2str(L*(1e3)), ' mm']);
disp(['Substrate Width:',num2str(Wsub*(1e3)), ' mm']);
disp(['Substrate Length:',num2str(Lsub*(1e3)), ' mm']);
disp(['Inset Length:',num2str(yo*(1e3)), ' mm']);
disp(['Patch separation:',num2str(a*(1e3)), ' mm']);


\end{lstlisting}